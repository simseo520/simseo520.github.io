\documentclass[11pt,a4paper]{article}
\usepackage[utf8]{inputenc}
\usepackage[margin=0.75in]{geometry}
\usepackage{enumitem}
\usepackage{titlesec}
\usepackage{hyperref}
\usepackage{setspace}

% Formatting
\pagestyle{empty}
\setlength{\parindent}{0pt}
\setlength{\parskip}{0.3em}
\titleformat{\section}{\normalsize\bfseries}{\thesection}{0em}{}[\titlerule]
\titlespacing*{\section}{0pt}{8pt}{4pt}

\begin{document}
\pagestyle{plain}

\singlespacing

% Header
\begin{center}
    {\Large \textbf{Teaching Statement}} \\
    \vspace{1mm}
    Seongbo Sim --- Indiana University, Bloomington
\end{center}

\vspace{3mm}

\section*{Teaching Philosophy}

My teaching centers on building conceptual understanding through theoretical rigor, empirical applications, and computational practice. I emphasize three core components: (1) \textbf{theoretical foundations} that provide frameworks for understanding economic phenomena; (2) \textbf{empirical applications} that bridge abstract models and observed data; and (3) \textbf{computational skills} essential for modern economic analysis. I connect classroom material to contemporary issues—the Global Financial Crisis, unconventional monetary policy, post-pandemic inflation—demonstrating how economic theory informs real-world policy debates. My goal is to equip students with analytical tools and empirical methods for careers in academia, policy, or industry.

\section*{Teaching Experience}

I have various teaching experience across institutions and levels. At Indiana University, I served as \textbf{Associate Instructor} for Statistical Analysis for Business and Economics (Spring 2023--Spring 2024), independently teaching 30+ students per semester. I designed lectures, created assessments, and introduced fundamental statistical concepts with Excel applications. As \textbf{Teaching Assistant} at Indiana University (seven semesters, 2019--2022, 2025), I supported Statistics and Fundamentals of Economics courses. At Yonsei University (2014--2015), I assisted with graduate Macroeconomics, undergraduate and graduate International Finance, and Economics of Money and Finance, gaining experience with diverse student populations and pedagogical approaches.

\section*{Teaching Methods}

I employ \textbf{active learning} and \textbf{incremental skill development}. Each topic begins with clear objectives and economic motivation. I build models step-by-step with visual aids, then incorporate problem-solving sessions where students work individually or in groups. For empirical courses, I use hands-on computational work with software (Excel, Matlab, Python), providing annotated code and replication exercises. I create inclusive environments through regular office hours, timely feedback, and adaptive pacing based on student needs.

\section*{Courses I Am Prepared to Teach}

\textbf{Undergraduate:} Principles/Intermediate Macroeconomics, Money and Banking, Introduction to Statistics.\\\textbf{Graduate:} Macroeconomic Theory, Macroeconometrics.

\section*{Sample Course: Undergraduate Money and Banking}

\textbf{Learning Objectives:} Understand money functions and interest rate determination; analyze central bank policy tools and transmission mechanisms; evaluate conventional and unconventional monetary policy; assess financial crises and regulation.

\textbf{Topics} (15 weeks): (1--2) Financial markets, interest rates, bond pricing; (3--4) Banking, intermediation, asymmetric information; (5--6) Money supply/demand, monetary frameworks; (7) Midterm; (8--9) Monetary policy tools and transmission; (10--11) Unconventional policy, QE, negative rates; (12--13) Financial crises (Great Depression, 2008); (14) Regulation, digital currencies; (15) Review/Final.

\textbf{Assessment:} Problem Sets 25\%, Midterm 25\%, Policy Brief 20\%, Final 25\%, Participation 5\%. \textbf{Text:} Mishkin, \textit{Economics of Money, Banking, and Financial Markets}. Supplemented with Fed publications, case studies, FRED data exercises.

\section*{Sample Course: Graduate Macroeconomic Theory I}

\textbf{Course Goals:} This course teaches students to formulate macroeconomic models and define equilibrium concepts. Students learn to analyze key economic issues through optimization in both sequential and recursive settings. Topics include growth theory, optimal fiscal policy, search models, monetary policy frameworks with incomplete markets, and DSGE models with unconventional policy tools.

\textbf{Prerequisites:} First-year macro and microeconomics sequences. Familiarity with dynamic optimization and basic programming recommended.

\textbf{Course Outline:} \textit{Sequential Methods}—Solow growth model, general equilibrium theory, neoclassical growth model, fiscal policy in sequential settings. \textit{Recursive Methods}—Dynamic programming, recursive formulation of neoclassical growth, search and matching models. \textit{Stochastic Problems} —Introduction to monetary policy frameworks with incomplete markets. (If time permits) Unconventional monetary policy in dynamic stochastic general equilibrium (DSGE) models.

\textbf{Textbooks:} Acemoglu, \textit{Introduction to Modern Economic Growth} (2009); Ljungqvist \& Sargent, \textit{Recursive Macroeconomic Theory} (2012); Walsh, \textit{Monetary Theory and Policy} (2017); Gali, \textit{Monetary Policy, Inflation, and the Business Cycle} (2015). Lecture notes posted on course site before each class, heavily based on textbooks—students encouraged to read relevant sections before/after lectures.

\textbf{Assessment:} Weekly problem sets 20\%, Midterm 30\%, Final 50\%. Problem sets graded on 1--3 scale (3 = strong understanding; 1 = needs review). Students may collaborate on problems but must submit individual solutions. All previously covered material fair game for exams.

\textbf{Course Philosophy:} Emphasis on developing technical skills for modern macroeconomic analysis. Students master both analytical and computational approaches, preparing them for advanced research in growth, fiscal policy, and dynamic macroeconomics.

\section*{Research Integration and Student Development}

My research on financial intermediation, monetary policy, and DSGE estimation directly informs teaching. For example, my job market paper on household deposit outflows provides contemporary case studies for money and banking courses, while my work on natural rate estimation demonstrates state-of-the-art Bayesian methods. I expose students to frontier research questions and show how coursework translates to meaningful contributions.

Beyond the classroom, I mentor students on research projects, career planning, and graduate applications. I actively seek feedback through evaluations and adapt methods based on best practices. Teaching is central to my professional identity—I am committed to creating engaging, rigorous learning environments and mentoring the next generation of economists.

\end{document}
