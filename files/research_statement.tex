\documentclass[11pt,a4paper]{article}
\usepackage[utf8]{inputenc}
\usepackage[margin=0.75in]{geometry}
\usepackage{enumitem}
\usepackage{titlesec}
\usepackage{hyperref}
\usepackage{setspace}

% Formatting
\pagestyle{empty}
\setlength{\parindent}{0pt}
\setlength{\parskip}{0.3em}
\titleformat{\section}{\normalsize\bfseries}{\thesection}{0em}{}[\titlerule]
\titlespacing*{\section}{0pt}{8pt}{4pt}

\begin{document}
\pagestyle{plain}

\singlespacing

% Header
\begin{center}
    {\Large \textbf{Research Statement}} \\
    \vspace{1mm}
    Seongbo Sim --- Indiana University, Bloomington
\end{center}

\vspace{3mm}

\doublespacing

\section*{Overview}

My research focuses on financial intermediation and monetary policy using structural DSGE models and Bayesian econometric methods. I address policy-relevant questions about financial stability, monetary transmission, and natural rate estimation. My work consists of two strands: (1) \textbf{household portfolio dynamics and banking sector stability}—examining the reallocation effects of the shock triggered by household deposit outflows; (2) \textbf{measurement of macroeconomic fundamentals}—developing methods to jointly estimate time-varying natural rates for policy decisions.

\section*{Job Market Paper}

\textbf{``Portfolio Shifts and Financial Intermediation: A DSGE Analysis of U.S. Household Deposit Outflows''}

U.S. household deposits declined 19.7\% since 2022—an unprecedented portfolio shift with significant implications for financial intermediation and monetary policy. I develop a New Keynesian DSGE model incorporating three key features: (1) declining non-transaction deposits, (2) increased bank holding company investment in subsidiary banks, and (3) changes in the distribution in credit supply.

Calibrated and estimated using U.S. data (2009:Q3--2025:Q2), the model shows that household financial distress sequentially alters asset composition, bank funding sources, and lending activities. This framework provides a tractable tool to analyze financial market reallocations and their policy implications for bank stability and monetary transmission. The research contributes to literature on household finance, financial intermediation, and monetary policy by structurally modeling contemporary financial developments.

\section*{Other Research}

\textbf{``A Holistic Approach to Macroeconomic Fundamentals: Joint Estimates of Natural Rates''} (with Regis Barnichon, Christian Matthes, and Byung Goog Park). 

We develop a method to jointly estimate natural rates (r*, u*, $\pi$*) from long-run macroeconomic data. Our approach embeds prior information into a time-varying parameter VAR with stochastic volatility, explicitly accounting for measurement error and outliers. This makes it well-suited for historical analysis including COVID-19 and post-pandemic inflation. Joint estimation captures important co-movements, providing more reliable estimates than univariate methods—directly informing central bank policy decisions.

\textbf{Third-Year Paper: ``Bayesian Estimation of the Four Equations New Keynesian Model.''} 

I investigate Quantitative Easing effects using the model of Sims et al. (2023) with market segmentation and financial frictions. Extending the model with habit formation and estimating via Bayesian methods on post-GFC data, I find QE has substantial expansionary effects with modest inflation. An endogenous QE rule is necessary to properly represent unconventional monetary policy.

\section*{Future Research Agenda}

My research will continue exploring financial intermediation, monetary policy, and macroeconomic dynamics through several directions:

\begin{itemize}
    \item \textbf{Financial Stability and Monetary Policy.} Extend the deposit outflows framework by allowing the portfolio decision by households to be influenced by monetary policy, analyzing feedback effects between bank stability and policy actions.    
    \item \textbf{Household Heterogeneity and Financial Decisions.} Incorporate heterogeneity in households portfolio choices depending on their wealth to the deposit outflows model, studying distributional effects on financial intermediation and monetary policy. As the wealthy households are the main driver of the recent deposit outflows in the U.S., this extension is crucial.
    \item \textbf{Integration of Financial Markets with Heterogeneous Agents.} Explore the process of integrating different financial markets with heterogeneous agents. The heterogeneity in financial literacy and system is the focus of this research. The goal is to provide a unified framework for the policy maker to forecast the effects of financial market integration between countries with varying degrees of financial development.
\end{itemize}


\section*{Methods and Policy Relevance}

My research methods involve combining tractable DSGE models with Bayesian estimation and time-varying parameter VARs. Proficient in Matlab, my training in structural modeling and empirical methods positions me to contribute across theoretical and applied research. My work maintains strong policy focus—deposit outflows research addresses bank supervisor concerns, while natural rate estimation provides tools for central banks to calibrate policy. I aim to produce research advancing both academic knowledge and practical policymaking.

\end{document}
