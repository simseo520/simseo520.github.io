\documentclass[11pt,a4paper]{article}
\usepackage[utf8]{inputenc}
\usepackage[margin=1in]{geometry}
\usepackage{enumitem}
\usepackage{titlesec}
\usepackage{hyperref}
\usepackage{setspace}

% Formatting
\pagestyle{empty}
\setlength{\parindent}{0pt}
\titleformat{\section}{\large\bfseries}{\thesection}{0em}{}[\titlerule]
\titlespacing*{\section}{0pt}{12pt}{6pt}

\begin{document}
\pagestyle{plain}


\onehalfspacing

% Header
\begin{center}
    {\LARGE \textbf{Seongbo Sim}} \\
    \vspace{2mm}
    Email: simseo@iu.edu \quad Phone: +82 10-2078-8620 \\
    Address: Wylie Hall, Indiana University, \\ 100 South Woodlawn Avenue, Bloomington, IN 47405 \\    
    Website: \url{https://simseo520.github.io/}
\end{center}

\vspace{5mm}

\section*{Education}
\textbf{Ph.D. in Economics} \hfill Expected Summer 2026 \\
Indiana University, Bloomington, IN \hfill GPA: 3.91/4.0 \\
Committee: Christian Matthes (main advisor), Todd B. Walker (chair), \\\phantom{Committee:}  Yoosoon Chang and Rupal Kamdar \\

\textbf{M.A. in Economics} \hfill 2017 \\
Yonsei University, Seoul, South Korea \hfill GPA: 3.94/4.0 \\
Dissertation: “The Determinants of the Benchmark Interest Rates of Bank of Korea\\
\phantom{Dissertation:} -- {\footnotesize Interest Rate Setting Behavior under Inflation Targeting Regime}” \\
Committee: Jung Sik Kim (chair), Taeyoon Sung and Kwang Hwan Kim\\

\textbf{B.A. in Economics and Philosophy (double major)} \hfill 2013 \\
Yonsei University, Seoul, South Korea \hfill GPA: 3.68/4.0\\
\emph{High Honors}\\

\section*{Research Interests}
Macroeconomics, Macroeconometrics, Financial Economics, Monetary Economics

\vspace{5mm}

\section*{Working Papers} 
\begin{itemize}[leftmargin=*]
    \item ``Portfolio Shifts and Financial Intermediation: A DSGE Analysis of U.S. Household Deposit Outflows'' (job market paper).
    \begin{itemize}
        \item \textbf{Abstract}: U.S. household deposits have declined by 19.7\% since 2022, representing an unprecedented shift in portfolio allocation with potentially significant implications for financial intermediation and monetary policy transmission. This paper develops a New Keynesian Dynamic Stochastic General Equilibrium (NK--DSGE) model with financial intermediation to investigate the effects of changes on the overall economy. The model incorporates stylized features to represent recent observations in the U.S. financial market, such as the decrease in non-transaction deposits held by households, the increase in investment from bank holding companies to their subsidiary banks, and the rise in loans from banks to non-bank financial institutions (NBFIs). The model is calibrated and estimated using U.S. data from 2009:Q3 to 2025:Q2. The model suggests that changes in the financial distress of households can alter the composition of their financial assets, the funding sources of banks, and their lending activities sequentially. The stylized framework of this paper provides a tractable tool to represent the reallocations in the financial market and investigate their implications for macroeconomic policies.
    \end{itemize}        
    \item ``A Holistic Approach to Macroeconomic Fundamentals: Joint Estimates of Natural Rates'' (with Regis Barnichon, Christian Matthes and Byung Goog Park).    
    \begin{itemize}
        \item \textbf{Abstract}: We develop a method to jointly estimate natural rates—or 'stars'—from long-run macroeconomic data. The approach embeds prior information about natural rates into a time-varying parameter VAR with stochastic volatility. It explicitly accounts for measurement error and outliers, making it well suited for historical analysis, including episodes like the COVID-19 pandemic and the post-pandemic inflation surge.
    \end{itemize}
\end{itemize}
\vspace{3mm}


\section*{Awards and Honors}
\begin{itemize}[leftmargin=*]   
\item Economic Distinguished Alumni Fellowship, Indiana University, 2023--2024.    
\item National Humanities and Social Graduate Research Scholarship, Korea Student Aid Foundation, 2015
\item Teaching Assistant Scholarship, Yonsei University, Fall 2014, Spring 2015
\item Brain Korea21 Research Assistantship, National Research Foundation of Korea, Fall 2014, Spring 2015
\item Graduate School of Economics Chair Professor Assistant Scholarship, Yonsei University, Fall 2013
\begin{itemize}
    \item Assisted Chair Professor Jun Kwang-woo.
\end{itemize}
\item University Designated Scholarship(Truth), Yonsei University, Fall 2010, Fall 2011
\end{itemize}

\vspace{5mm}

\section*{Conference Presentations}
\begin{itemize}[leftmargin=*]
    \item Western Economic Association International, The 90th Annual Conference, Honolulu, HI (2015)
    \begin{itemize}
        \item \textbf{Title}: Interest Rate Setting Behavior for Korea: Role of External Factors 
        \item Co-authored with Jung-Sik Kim and Young-Ki Lee
    \end{itemize}    
\end{itemize}
\vspace{5mm}

\section*{Research Experience}
\textbf{Research Assistant} 
\begin{itemize}[leftmargin=*]
    \item Christian Matthes \hfill Summer 2021
    \begin{itemize}
        \item Indiana University, Bloomington (currently at the University of Notre Dame)
        \item Checked Matlab codes for the project on daily inflation.    
    \end{itemize}
    \item Jungsik Kim \hfill 2014
    \begin{itemize}
        \item Yonsei University, Seoul.       
        \item Assisted the data analysis for the paper ``Effects of Capital Gain Taxes on Revenue and Stock Trade Volume in Korea,'' published in \emph{The Korea International Economic Association} (2014).
    \end{itemize}
\end{itemize}

\vspace{5mm}


\section*{Teaching Experience}
\textbf{Associate Instructor}
\begin{itemize}[leftmargin=*]   
    \item Statistical Analysis for Business and Economics
    \begin{itemize}[leftmargin=*]
        \item Indiana University, Spring 2023 -- Spring 2024.
        \item 30+ students per semester.
        \item The course is for undergraduate students majoring in Business who learn basic statistical concepts and applications using Excel.
    \end{itemize}
\end{itemize}

\vspace{5mm}
\textbf{Graduate \& Teaching Assistant} 
\begin{itemize}[leftmargin=*]   
\item Statistical Analysis for Business and Economics
    \begin{itemize}[leftmargin=*]
        \item Indiana University, Fall 2019, Spring 2020, Spring 2021, Fall 2021, Spring 2022, Fall 2022, Spring 2025.
        \item Assisted Senior Lecturer Nastassia Krukava.
        \item The course is for undergraduate students majoring in Business who learn basic statistical concepts and applications using Excel.
    \end{itemize}    
\item Fundamental of Economics for Business I
    \begin{itemize}[leftmargin=*]
        \item Indiana University, Fall 2020.
        \item Assisted Teaching Professor Paul Graf.
        \item The course is for undergraduate students majoring in Business who learn basic microeconomic concepts and applications.
    \end{itemize}    
\item Macroeconomics B
    \begin{itemize}[leftmargin=*]
        \item Yonsei University, Fall 2014, Spring 2015, Fall 2015
        \item Assisted Professor Jung Sik Kim.
        \item The course is for students of Graduate School of Economics who learn intermediate macroeconomic theories and applications.
    \end{itemize}
\item International Finance
    \begin{itemize}[leftmargin=*]
        \item Yonsei University, Fall 2014, Spring 2015, Fall 2015
        \item Assisted Professor Jung Sik Kim.
        \item The course is for undergraduate students who learn basic international financial theories and applications.
    \end{itemize}
\item International Finance
    \begin{itemize}[leftmargin=*]
        \item Yonsei University, Fall 2015
        \item Assisted Professor Jung Sik Kim.
        \item The course is for graduate students majoring in Economics who learn advanced international financial theories and applications.
    \end{itemize}    
\item Economics of Money and Finance
    \begin{itemize}[leftmargin=*]
        \item Yonsei University, Fall 2014, Fall 2015
        \item Assisted Professor Jung Sik Kim.
        \item The course is for undergraduate students who learn basic monetary and financial theories and applications.
    \end{itemize}
\end{itemize}

\vspace{5mm}


\section*{List of Coursework in Ph.D. Program}
\begin{itemize}[leftmargin=*, align=left]  
    \raggedright
    \item \textbf{Microeconomic Theory:} Optimization Theory for Economic Analysis, \\\phantom{\textbf{Microeconomic Theory:}} Theory of Prices and Markets I \& II, Game Theory
    \item \textbf{Macroeconomic Theory:} Macroeconomic Theory I \& II, Seminar in Money
    \item \textbf{Econometrics:} Econometrics I-III,  Microeconometrics
    \item \textbf{Macroeconometrics:} Macroeconometrics, Empirical Macro I \& II
    \item \textbf{Finance:} Asset Pricing Theory, Empirical Finance
    \item \textbf{Statistics \& Computing:} Bayesian Theory \& Data Analysis, Machine Learning,\\ \phantom{\textbf{Statistics \& Computing:}} Applied Machine Learning
    \item \textbf{Advanced Seminars:} Advanced Macro Seminar, Network Formation Games, \\ \phantom{\textbf{Advanced Seminars:}} Advanced Econometrics Seminar
    \item \textbf{Other:} Teaching Undergraduate Economics
\end{itemize}
\vspace{5mm}

\section*{Requirements for Ph.D. Program}
\begin{itemize}[leftmargin=*]
    \item Obtained Ph.D. Candidacy \hfill January 2023
    \item High passed on Third year paper \hfill Summer 2022
    \begin{itemize}
        \item \textbf{Title}: ``Bayesian Estimation of the Four Equations New Keynesian Model''
        \item \textbf{Committee}: Christian Matthes (chair), Yoosoon Chang, Laura Liu.
        \item \textbf{Abstract}: 
        This paper investigates the effects of “Quantitative Easing” using the four-equation New Keynesian model of Sims et al. (2023), which features market segmentation and financial frictions. The model is estimated on macroeconomic data covering the period after the Global Financial Crisis and represents a stylized economy in which the monetary authority
        can reduce the policy rate into negative territory to stimulate activity. As a new feature, I introduce habit formation in households’ consumption. Habit formation helps capture short-run fluctuations of macroeconomic variables and generates additional persistence. Sims et al. (2023) show the expansionary effects of quantitative easing with relatively modest inflation using calibration. The Bayesian estimation of a modified model produces similar results and indicates a more substantial role for quantitative easing. I also find that an endogenous quantitative easing rule is necessary to represent the role of quantitative easing properly.
    \end{itemize}
    \item High passed Macro Theory (Empirical) \hfill Summer 2021
    \item High passed Micro/Macro Core Theory Exams \hfill Summer 2020 
\end{itemize}

\vspace{5mm}
\section*{Skills}
\textbf{Software:} Matlab, LaTeX, Python \\
\textbf{Languages:} Korean (Native), English (Proficient)

\end{document}