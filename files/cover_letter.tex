\documentclass[11pt,a4paper]{article}
\usepackage[utf8]{inputenc}
\usepackage[margin=1in]{geometry}
\usepackage{hyperref}

% Formatting
\pagestyle{empty}
\setlength{\parindent}{0pt}
\setlength{\parskip}{0.8em}

\begin{document}

\begin{flushleft}
\textbf{Seongbo Sim} \\
Department of Economics, Indiana University \\
Wylie Hall, 100 South Woodlawn Avenue, Bloomington, IN 47405 \\
Email: simseo@iu.edu | Phone: +82 10-2078-8620 | \url{https://simseo520.github.io/}
\end{flushleft}

\vspace{0.5em}

\today

\vspace{0.5em}

\begin{flushleft}
{[}Hiring Committee Chair Name{]} \\
{[}Department Name{]} \\
{[}University Name{]} \\
{[}Address Line 1{]} \\
{[}Address Line 2{]}
\end{flushleft}

\vspace{1em}

Dear Members of the Search Committee,

I am writing to apply for the {[}Position Title{]} position in the {[}Department Name{]} at {[}University Name{]}. I am a PhD candidate in Economics at Indiana University, expecting to graduate in Summer 2026. My research focuses on macroeconomics, macroeconometrics, and financial economics.

My dissertation, supervised by Christian Matthes, Todd B. Walker, Yoosoon Chang, and Rupal Kamdar, examines financial intermediation and household portfolio dynamics in monetary policy transmission. My job market paper, ``Portfolio Shifts and Financial Intermediation: A DSGE Analysis of U.S. Household Deposit Outflows,'' develops a New Keynesian DSGE model with a segmented financial sector to study U.S. households' post-pandemic shift out of non-transaction deposits. Households allocate savings between deposits and bank-holding-company debt subject to financial-distress costs; BHC equity capital replaces lost deposits; retail banks extend credit lines; and a bank-funded shadow bank lends to riskier startups. Calibrated and Bayesian-estimated on U.S. data (2009:Q3--2025:Q2), the model simulates a 24\% decline in deposits from 2022:Q1 to 2025:Q1, matching the observed 20.4\% decline. A household financial-distress shock reduces deposit shares, increases BHC investment, and reallocates credit away from traditional bank borrowers toward Nonbank Financial Institutions.

My second paper, ``A Holistic Approach to Macroeconomic Fundamentals: Joint Estimates of Natural Rates'' (with Regis Barnichon, Christian Matthes, and Byung Goog Park), develops a method to jointly estimate natural rates from long-run macroeconomic data. The approach embeds prior information into a time-varying parameter VAR with stochastic volatility, explicitly accounting for measurement error and outliers, making it well suited for historical analysis including the COVID-19 pandemic and post-pandemic inflation surge. I have substantial teaching experience at Indiana University as Associate Instructor for Statistical Analysis for Business and Economics (2023--2024) and Teaching Assistant for seven semesters (2019--2022, 2025), as well as at Yonsei University (2014--2015). I am prepared to teach courses in macroeconomics, monetary economics, financial economics, and macroeconometrics.

I am excited about the opportunity to join {[}Department Name{]} at {[}University Name{]}, where your department's strength in {[}specific research area/faculty members{]} aligns well with my research interests. My future research will continue exploring financial intermediation, monetary policy, and macroeconomic dynamics. Thank you for considering my application. I have enclosed my curriculum vitae, research statement, teaching statement, job market paper, and letters of recommendation.

Sincerely,

\vspace{1em}

Seongbo Sim

\end{document}
